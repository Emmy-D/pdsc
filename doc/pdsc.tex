\documentclass[twocolumn]{extarticle}
\usepackage{fullpage}
\usepackage{amssymb}
\usepackage{sidecap}
\usepackage{units}
\usepackage{amsmath}
\usepackage{graphicx}
\usepackage{subfig}
\usepackage{hyperref}
\usepackage[margin=0.95in]{geometry}
\usepackage[all]{hypcap}
\usepackage{units}
\usepackage{upgreek}
\usepackage{acronym}

\hypersetup{ %
  pdftitle={},
  pdfauthor={},
  pdfkeywords={},
  pdfborder=0 0 0,
  pdfpagemode=UseNone,
  colorlinks=true,
  linkcolor=black,
  citecolor=black,
  filecolor=black,
  urlcolor=black,
  pdfview=FitH}

\acrodef{pds}[PDS]{Planetary Data System}
\acrodef{usgs}[USGS]{United States Geological Survey}
\acrodef{pilot}[PILOT]{Planetary Image Locator Tool}
\acrodef{pdsc}[PDSC]{PDS Coincidences}
\acrodef{sql}[SQL]{Structured Query Language}

\sidecaptionvpos{figure}{t}

\makeatletter
\renewcommand\@maketitle{%
\noindent\begin{minipage}{0.95\textwidth}
%\vskip 0.7em
\let\footnote\thanks 
{\LARGE \@title \par }
\vskip 0.5em
{\large \@author \par}
\end{minipage}
\hfill
\vskip 2.0em \par
}
\makeatother

\title{
  PDSC: Planetary Data System Coincidences
}
\author{Gary Doran, Kiri Wagstaff, and Lukas Mandrake}
\date{\today}

\begin{document}
\maketitle

\section*{Overview}
The \ac{pds} contains a wealth of publicly available information collected by
spacecraft from all across the solar system. In addition to hosting this
information, the \ac{pds} also provides capabilities to search through data
products by spacecraft, instrument, or even image content. However, several
other types of queries are also useful for those requiring cross-instrument
comparison or the comparison of images taken of the same region at different
times. Furthermore, users might desire the capability to query sets of
overlapping images programmatically, without relying on manually using a web
interface.

\section*{Solution}
To support queries of the type described above, we have developed a Python
library called \ac{pdsc}. The library ingests cumulative index files of
spacecraft imaging the surface of Mars from orbit, and allows querying images
(1) coincident with a given latitude/longitude location to within some
specified radius, or (2) overlapping some other image. The library is designed
to be extensible to other instrument types, and it could readily be extended to
handle images of planetary bodies other than Mars. The library is designed to
be fast, responding to most queries in less than a second.

\section*{Description}

\subsection*{Ingestion}
The \ac{pdsc} library allows for quickly querying coincident observations by
constructing a set of special databases and index structures during the
ingestion process. The first database constructed is simply a \ac{sql} version
of the PDS cumulative index. The second set of index structures holds geometric
information used for coincidence queries.

The footprints of observations on the surface of Mars are reported differently
across instruments. Some instruments report four corners of the observation in
latitude and longitude coordinates. Others report a latitude/longitude
location at the center of the observation, along with a ``north azimuth'' (the
direction that the ground track makes with north. There are several challenges
to using this information for determining which points fall within the
observation footprint. First, there might be errors in geo-registration, so that
the observation location is not entirely accurate. We do not attempt to solve
this challenge. The second challenge is that it takes more than four corners to
specify the precise boundary of an observation, since there is ambiguity over
how to connect such corners over long distances. For example, for long tracks,
the footprint is not bounded by geodesic lines. However, they are also not
connected by ``rhumb'' lines of constant bearing, especially over polar regions.

Thus, accurately describing a footprint requires projecting a geodesic line of
flight on the surface of the body, then moving perpendicularly to the line of
flight to determine the location of cross-track pixels. This method can be used
to decompose the observation footprint into smaller polygonal segments, which
are sufficiently small as to allow assuming that their vertices are connected by
simple geodesic lines. For reasons described later, we use triangular segments
to ensure that the segments are \emph{convex} polygons, a property that
simplifies the querying process.

These triangular segments are placed into a \ac{sql} database that records the
observation to which they belong along with the vertices that define their
corners. One final data structure saved to disk is a \emph{ball tree}, which is
populated with the set of segment centers. The radius (maximum distance from a
segment vertex to its center) is also recorded along with the tree. The use for
the ball tree, which allows querying all segment centers that fall within some
radius of a given query point, is also described below.

\subsection*{Queries}

There are two major query types supported by \ac{pdsc}. The first type of query
finds all observations within some distance $\epsilon$ of a given point on the
surface of Mars. The second finds all observations that overlap with another
observation. We discuss these two query types below.

\subsubsection*{Point Queries}

For point queries, a user would like to find all observations within $\epsilon$
of a query point.  To start, consider the case where $\epsilon = 0$, so the
query point must be \emph{within} the observation. To avoid having to check
point inclusion within every single observation, a pre-filtering step supported
by the index structures is performed. If a point falls within an observation,
then it falls within \emph{some} polygonal segment into which the observation
has been decomposed. Thus, we can reduce the problem to finding point inclusion
within triangular segments. Furthermore, if a point falls within a triangular
segment, then it is within radius $r$ from the segment center, where $r$ is the
largest distance from the center of the triangle to any of its vertices.

We use the ball tree index to quickly find all segments whose centers are at
most radius $r$ from the query point. This operation takes $O(\log n)$ time,
where $n$ is the number of segments in the database. Thus, we can significantly
reduce query time from the $O(n)$ operations needed to exhaustively search all
segments. After this pre-filtering step, we are left with a much smaller set of
segments to exhaustively check for point inclusion.

To check for point inclusion within a segment, we use several properties of the
generated segmentation. First, we assume that the segments are small enough that
the vertices of the segments are connected via geodesic lines. By definition, a
geodesic is the intersection of a sphere with a plane passing through the
center of the sphere. To avoid working in spherical coordinates, we explicitly
map the point on the surface to a 3--D point on a unit sphere. For each edge
that bounds the segment, we compute a 3--D plane passing through the two
vertices of the edge (also represented on a unit sphere) and the center of the
sphere). As long as each edge of the segment has nonzero length,
the edge points $(e_1, e_2)$ and sphere center $c$ uniquely determine a plane,
whose normal vector can be found using the cross product $(e_1 - c)\times (e_2 -
c)$. If segment vertices are traversed in counter-clockwise order, then by the
right-hand rule, the plane normal vectors computed in this fashion will point
towards the center of the polygonal segment.

Then, because triangular segments are guaranteed to be convex, to see
if the point falls within the polygon, it suffices to check that the point is on
the correct (positive) side of each bounding plane. This is easily computed with
a dot product between the point and the plane normal vectors. After we have
determined that a point falls within an observation, we do not need to continue
checking inclusion within any segment that is also part of that observation.

Now, suppose we have the case where $\epsilon > 0$; that is, we want to see if a
point is at most $\epsilon$ distance away from an observation footprint. The
approach follows that above with some minor changes. First, we must relax the
ball tree query to look for segments whose centers are within $r + \epsilon$ of
the query point. The triangle inequality guarantees that for points within
$\epsilon$ of a segment, the segment center must be at most $r + \epsilon$ away
from the point.

Next, rather than compute point inclusion, we must compute the distance between
the query point and the segment. We first check for inclusion, in which case
the distance is zero. Otherwise, we know that the point lies outside the
polygon. Thus, the distance between the point and the polygon is the shortest
distance between the query point and any point on the boundary of the polygon.
This boundary point is either on an edge or a vertex. The closest point to the
query on an edge is approximately the query point in 3--D projected onto the
plane passing through the edge and the center of the unit sphere (the same plane
used for inclusion tests above), then projected back out to the surface of the
unit sphere. If we perform this projection and the projected point falls within
the polygon (strictly speaking, on the boundary of the polygon), then this point
is a candidate for the closest point in the polygon to the query. Otherwise, if
the projection falls outside the polygon, then it is not considered as a
candidate (i.e., some vertex is closer to the query point).

By performing the procedure above for each edge, we have a set of up to three
candidates for the closest point in the polygon to the query point. We also
include the three vertices as candidates. These points are all in spherical
coordinates. Then, using the haversine formula, we compute the geodesic distance
between each candidate and the query point. The minimum of these distances is
the shortest distance between the query and the polygon. If this distance is
less than $\epsilon$, then the observation satisfies the query and is returned.
As for the inclusion check, if any segment from an observation satisfies the
query, the remaining segments from that observation do not need to be checked.

\subsubsection*{Overlap Queries}

The second type of query supported by \ac{pdsc} enables users to find all
observations (of any supported instrument) that overlap a given query
observation. To support this query, a similar approach to the point query is
used. First, the pre-computed query observation segmentation is fetched from the
\ac{sql} database containing segments. Two observations overlap if and only if
at least one pair of segments across the pairs overlap. Thus, for each query
observation segment $s_i$, we must see which other segments in the database
overlap $s_i$.

If two segments $s_i$ and $s_j$ overlap, then there must be some point that lies
within each segment. If these segments have radii $r_i$ and $r_j$, then this
means that by the triangle inequality, the segment centers are at most $r_i +
r_j$ apart from each other. Thus, if $c_i$ is the center of the query segment
with radius $r_i$, and $r$ is the maximum segment radius for an instrument, then
the ball tree is queried with point $c_i$ and radius $r_i + r$ to find all
segments that possibly overlap with each query segment. This is done for every
segment in the query observation.

For a pair of candidate overlapping segments, checking for overlap is performed
in two dimensions by first projecting the segments onto a common plane. The
plane is chosen to be tangential to the surface to the sphere on which the
segments lie at a point equidistant between the segment centers. The selection
of this plane is intended to minimize the distortions caused by projecting the
polygons onto a plane.

The process above is performed for each pair of candidate overlapping segments.
As soon as some pair of overlapping segments across two observations is found,
the remaining pairs of segments do not need to be checked.

\section*{Novelty}

Several exiting tools allow limited capabilities for searching \ac{pds} images.
For example, the \ac{pds} Image
Atlas\footnote{\url{https://pds-imaging.jpl.nasa.gov/search/}} enables image
search by metadata or even content, but not by point inclusion or overlap. The
\ac{usgs} maintains its \ac{pilot}
database,\footnote{\url{https://pilot.wr.usgs.gov/}} which does enable some
search using image location. In particular, one can query to find overlapping
stereo pairs of images from the same instrument. However, the search does not
enable searching for overlapping pairs across images. Finally,
JMARS\footnote{\url{https://jmars.asu.edu/}} provides a graphical interface for
querying \ac{pds} images base on location, but not overlapping pairs. The tools
mentioned above all rely primarily on a user interface component, whereas the
\ac{pdsc} library allows searching directly from Python, which could be
incorporated into a user interface if desired.

In addition to the novel search capabilities offered by \ac{pdsc}, it also uses
a new combination of existing database and data structure technologies to
improve search efficiency. In particular, the ball tree data structure
significantly improves the computation complexity of finding observation
segments from $O(n)$ to $O(\log n)$ time.

\bibliographystyle{plain}
\bibliography{bibliography}

\end{document}
